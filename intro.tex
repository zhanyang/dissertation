\chapter{Introduction}
\label{chap:intro}

Today's general purpose file systems fail to utilize the full bandwidth of the
underlying hardware.
Widely used inode-based file systems, such as ext4, XFS, and Btrfs, can write
large files at near disk bandwidth,
but typically create small files at less than 3\% of the disk bandwidth.
Similarly, these file systems can read large files at near disk bandwidth,
but scanning directories with many small files is slow, and the performance
degrades when the file system ages~\citep{betrfs3}.

At the heart of this issue is how data is organized on disk.
The most common desgin pattern fro modern file systems is to use multiple layers
of indirection.
The inode number connects the name of a file or directory entry in a directory
to its metadata location on disk.
The metadata of an inode contains extents that describes the physical location
and length of data at different offset.
Indirection simplifies implementation of the file system, and makes some
operations, such creating files and appending files, easy to implement.
In particular, namespace operations are simple and flexible.
For example, a rename is just a pointer swings, moving one entry from one
directory to another directory.
However, indirection doesn't impose any constraint on how metadata and data
are placed on the disk.
In the worst case, the metadata of entries under a directory and the content of
a file can end up scattered over the disk.
Heuristics, such as cylinder groups~\citep{ffs1}, are designed to mitigate this
problem.
However, on modern inode-based file systems, unless the metadata or data are
modified, their location doesn't change.
Therefore, after disk space is allocated and freed again and again, the free
space on disk becomes scattered and perfect placement for future metadata and
data becomes impossible, leading to bad performance when file system ages.

