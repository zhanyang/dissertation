\chapter{Conclusion}
\label{chap:conclusion}

File systems faces trade-offs between the performance of namespace
operations and spatial locality.
On one hand,
traditional inode-based file systems have good performance for namespace
operations because of the indirection between a directory and entries in the
directory.
However, this indirection stops these file systems from grouping metadat and data
under one directory close to each other on disk,
especially when the file system ages.
On the other hand, full-path-indexed file systems ensure locality by indexing
metadata and data by full-paths.
However, existing full-path-indexed file systems either have unbounded I/O costs for
namespace operations, or taxes other operations for efficient namespace operations.

This dissertation first presents a new operation on \bets, range-rename,
that updates all keys with one prefix to have another prefix.
The range-rename operation adopts \textbf{key lifting} to transform \bets into
lifted \bets
and accomplishes its task in a bounded number of I/Os through
\textbf{tree surgery}.
By invoking range-rename operations for file system renames,
full-path-indexed \betrfs has bounded I/O costs for its renames.

Then, this dissertation introduces another new operation on \bets, range-clone,
that clones all keys with one prefix to have another prefix.
By transforming lifted \bets into lifted \bedags, range-clone can utilize
the techniques introduced by range-rename to complete its work on the critical path.
Moreover, the range-clone operation can delay the tree surgery work with a new
type of messages, \goto messages, fitting itself into the write-optimized
framework of \bedags.
Full-path-indexed \betrfs thus has write-optimized file or directory renames and
clones by implementing them with range-clone operations.

This dissertation shows that with the right optimizations, a full-path-indexed,
write-optimized file system can
have both efficient namespace operations and locality.
There is no trade-off between them.
Also, full-path indexing opens up new opportunities for namespace operations,
such as directory clones.

