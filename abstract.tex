\begin{abstract}

Write-optimized file systems store metadata and data in write-optimized
dictionaries for good random write performance.
Prior write-optimized file systems also adopt full-path indexing for spacial
locality,
because full-path-indexing places metadata and data under the same directory
close to each other on disk.
However, prior works show that namespace operations can be prohibitively
expensive on full-path-indexed file systems.
For instance, renaming a directory requires prior file systems to update all
data in that directory.

This dissertation shows that full-path-indexed, write-optimized file systems can
implement namespace operations in an I/O-efficient way by directly operating on
the underlying write-optimized dictionaries.
In fact, full-path indexing creates more opportunities for efficient namespace
operations.
For instance, full-path-index file systems can clone a directory with a few
operations in the underlying write-optimized dictionaries,
without traversing all files in the directory.

\end{abstract}
