\usepackage{pgfplots}
\usepackage{tikz}

\newcommand{\betrfs}{BetrFS\xspace}
\newcommand{\betrfsOne}{BetrFS 0.1\xspace}
\newcommand{\betrfsTwo}{BetrFS 0.2\xspace}
\newcommand{\betrfsThree}{BetrFS 0.3\xspace}
\newcommand{\betrfsFour}{BetrFS 0.4\xspace}
\newcommand{\bet}{B$^{\varepsilon}$-tree\xspace}
\newcommand{\bets}{B$^{\varepsilon}$-trees\xspace}
\newcommand{\btree}{B-tree\xspace}
\newcommand{\btrees}{B-trees\xspace}
\newcommand{\fti}{\textit{ft-index}\xspace}
\newcommand{\Fti}{\textit{Ft-index}\xspace}
\newcommand{\klibc}{\textbf{klibc}\xspace}
\newcommand{\mdb}{\texttt{meta\_db}\xspace}
\newcommand{\ddb}{\texttt{data\_db}\xspace}
\newcommand{\spre}{\textit{src\_prefix}\xspace}
\newcommand{\dpre}{\textit{dst\_prefix}\xspace}
\newcommand{\goto}{\texttt{GOTO}\xspace}
\newcommand{\delold}{\textit{del\_old}\xspace}
\newcommand{\bedag}{B$^{\varepsilon}$-DAG\xspace}
\newcommand{\bedags}{B$^{\varepsilon}$-DAGs\xspace}

\pgfkeys{
    /fs-names/ext4/.initial=ext4,
    /fs-names/btrfs/.initial=Btrfs,
    /fs-names/xfs/.initial=XFS,
    /fs-names/zfs/.initial=ZFS,
    /fs-names/nilfs2/.initial=NILFS2,
    /fs-names/betrfs3/.initial=\betrfsThree,
    /fs-names/betrfs3-max/.initial=\betrfsThree with one zone,
}

\pgfkeys{
    /fs-colors/ext4/.initial=blue,
    /fs-colors/btrfs/.initial=red,
    /fs-colors/xfs/.initial=green,
    /fs-colors/zfs/.initial=purple,
    /fs-colors/nilfs2/.initial=cyan,
    /fs-colors/betrfs3/.initial=orange,
    /fs-colors/betrfs3-max/.initial=black,
}

\pgfkeys{
    /fs-marks/ext4/.initial=triangle*,
    /fs-marks/btrfs/.initial=pentagon*,
    /fs-marks/xfs/.initial=square*,
    /fs-marks/zfs/.initial=diamond*,
    /fs-marks/nilfs2/.initial=o,
    /fs-marks/betrfs3/.initial=oplus,
    /fs-marks/betrfs3-max/.initial=+,
}

\newcommand{\addTokubenchZonePlot}[1]
{
    \addplot[
        color=\pgfkeysvalueof{/fs-colors/#1},
        line width=0.75pt,
        mark=\pgfkeysvalueof{/fs-marks/#1},
    ]
    plot[
    ]
    table[
    ]
    {./data/tokuzone/#1.csv};
    \addlegendentry{\pgfkeysvalueof{/fs-names/#1}}
}

\newtheorem{invariant}{Invariant}
