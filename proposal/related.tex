\section{Related work}
\label{sec:related}

\subsection{Write-optimized file systems}

Most write-optimized file systems are built on FUSE~\cite{fuse} and
LSM-trees~\cite{lsm}.

KVFS~\cite{kvfs} is built on VT-trees, LSM-trees with stitching.
When VT-trees flush and merge two SSTables (similar to nodes in \bets), it
identities overlapping key ranges and only merges these ranges.
This technique, which is called stitching, avoids writing the same block
multiple times but causes fragmentation.
Therefore, KVFS needs to defragment when the disk is nearly full.

TableFS~\cite{tablefs} keeps metadata and small files (less than 4KB) on the
LSM-tree and large files in the underlying ext4 file system.
Therefore, TableFS outperforms other file systems on metadata and small file
benchmarks.

TokuFS~\cite{tokufs} is a full-path indexing file system with \bets on FUSE.
It is the precursor to \betrfs, showing good performance for small writes and
directory scans.

\subsection{File systems with snapshots}

Many modern file systems provide snapshot mechanism, making read-only copies of
the whole file system.

The WAFL file system~\cite{wafl} organizes all blocks in a tree structure.
By copying the vol\_info block, which is the root of the tree structure, WAFL
creates a snapshot.
Later, WAFL introduces a level of indirection between the file system and the
underlying disks~\cite{wafl-flexvol}.
Therefore, multiple active instances of the file system can exist at the same
time and WAFL can create writable snapshots of the whole file system by creating
a new instance and copying the vol\_info block.

FFS~\cite{ffs} creates snapshots by suspending all operations and creating a
snapshot file whose inode is stored in the superblock.
The size of the snapshot file is the same as the disk.
Upon creation, most block pointers in the snapshot inode are marked
``not copied'' or ``not used'' while some metadata blocks are copied to new
addresses.
Reading a ``not copied'' address in the snapshot file results in reading the
address on the disk.
When a ``not copied'' block is modified in the file system, FFS copies the block
to a new address and updates the block pointer in the snapshot inode.

NILFS~\cite{nilfs2} is a log-structured file system that organizes all blocks
in a B-tree.
In NILFS, each logical segment contains modified blocks and a checkpoint block
used as the root of the B-tree.
NILFS gets the current view of the file system from the checkpoint block
of the last logical segment.
NILFS can create a snapshot by making a checkpoint block permanent.

ZFS~\cite{zfs} also stores the file system in a tree structure and creates
snapshots by copying the root of the tree (uberblock).
To avoid maintaining one block bitmap for each snapshot, ZFS keeps birth time
in each pointer.
A block should not be freed if its birth time is earlier than any snapshot.
In that case, the block is added to the dead list of the most recent snapshot.
When a snapshot is deleted, all blocks in its dead list are checked again before
being freed.

GCTree~\cite{gctree} implements snapshots on top of ext4.
It chains different versions of a metadata block with GCTree metadata.
Also, each pointer in the metadata block has a ``borrowed bit'' indicating
whether the target block is inherited from the previous version.
Therefore, GCTree can check whether to free a block by inspecting GCTree
metadata and doesn't need to maintain reference counts.

NOVA-Fortis~\cite{nova} targets at non-volatile main memory (NVMM).
In NOVA-Fortis, each inode has a private log with log entries pointing
to data blocks.
To enable snapshots, NOVA-Fortis keeps a global snapshot ID and adds the
creating snapshot ID to log entries in inodes.
NOVA-Fortis decides whether to free a block based on the snapshot ID in the log
entry and active snapshots.
It also deals with DAX-style \texttt{mmap} by stalling page faults when marking
all pages read-only.

\subsection{File systems with file or directory clones}

Some file systems go futher to support cloning one single directory or file.

The Episode file system~\cite{episode} groups everything under a directory into
a fileset.
Episode can create an immutable fileset clone by copying all the anodes (inodes)
and marking all block pointers in the anodes copy-on-write.
When Episode changes a block with a copy-on-write pointer, it allocates a new
block and updates the block pointer in the active fileset.

Ext3cow~\cite{ext3cow} creates immutable file clones by maintaining a
system-wide epoch and inode epochs.
When an inode is modified, ext3cow allocates a new inode if the epoch in the
inode is older than the last snapshot epoch.
Also, each directory stores birth and death epochs for each entry.
Ext3cow can render the view of the file system in a certain epoch by fetching
entries alive at that epoch.

Btrfs~\cite{btrfs} supports creating writable snapshot of the whole file system
by copying the root of the sub-volume tree in its COW friendly
B-trees~\cite{cowbtree}.
Btrfs clones a file by sharing all extents of the file.
The extent allocation tree records extents with back pointers to the inodes.
Therefore, Btrfs is able to move an extent at a later time.

\subsection{Versioning file systems}

There are also versioning file systems that versions files and directories.

EFS~\cite{efs} automatically versions files and directories.
By allocating a new inode, it creates and finalizes a new file version when the
file is opened and closed.
Each versioned file has an inode log that keeps all versions of the file.
All entries in directories have creation and deletion time.

CVFS~\cite{cvfs} tries to store metadata versions more compactly.
CVFS suggests two ways to save space in versioning file systems:
1. journal-based metadata that keeps a current version and an undo log to
recover previous versions;
2. multiversion B-trees that keep all versions of metadata in a B-tree.

Versionfs~\cite{versionfs} builds a stackable versioning file system.
Instead of building a new file system, Versionfs adds the functionality of
versioning on an existing file system by transferring certain operations to
other operations.
In this way, all versions of a file are maintained as different files in the
underlying file system.
