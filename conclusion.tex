\chapter{Conclusion}
\label{chap:conclusion}

File systems faced trade-offs between the performance of namespace
operations and spatial locality.
Traditional inode-based file systems have good performance for namespace
operations because of the indirection between a directory and entries under it.
However, this indirection stops these file systems from grouping data under one
directory close to each other on disk,
especially when the file system ages.
On the other hand, full-path-indexed file systems ensure locality by indexing
metadata and data by full-paths.
However, existing full-path-indexed file systems either have unbounded cost for
namespace operations, or taxes other operations for good namespace operation
performance.

This dissertation first present a new operation on \bets, range-rename,
that updates all keys with one prefix to have another prefix.
The range-rename operation adopts \textbf{key lifting} to transform \bets into
lifted \bets
and accomplishes its task in a bounded number of IOs through
\textbf{tree surgery}.
By calling range-renames to complete file-system renames, full-path-indexed
\betrfs thus has bounded costs for its renames.

Then, this dissertation introduces another new operations on \bets, range-clone,
that clones all keys with on prefix to have another prefix.
By transforming lifted \bets into lifted \bedags, range-clone can utilize
range-rename techniques to perform its work on the critical path.
Moreover, the range-clone can delay the tree surgery work with a new type of
message, \goto messages, fitting into the write-optimized framework of \betrfs.
Full-path-indexed \betrfs thus has write-optimized file or directory renames and
clones by implementing them with range-clones.

This dissertation shows that with the right optimizations, a file system can
have both efficient namespace operations and locality.
There is no trade-off between them.

