\section{Introduction}
\label{sec:intro}

Inode-based file systems assign an inode number to a file or directory upon
creation and index the metadata and data of the file or directory using the
inode number.
Inodes add a level of indirection so that file systems can change the data of
one file or directory without affecting other files or directories.
However, this schema doesn't impose any constraint on the placement of metadata
and data under one directory.
Thus, in the worst case, the metadata and data can end up scattered over the
disk.

On the other hand, \betrfs uses full-path indexing on top of \bets.
Metadata and data are stored as key/value pairs whose keys are the full-paths of
the associated files and directories.
Therefore, metadata and data under one directory are contiguous in the keyspace.
Because of the large physical node size in \bets, most logically related
metadata and data are stored close to each other on disk.

However, namespace operations are the longstanding problem of full-path-indexed
file systems.
For example, a rename must update all the keys of key/value pairs being renamed.
A naive rename implementation would read all the old key/value pairs, insert
them with new keys and delete them.
When the size of the related key/value pairs is huge, such rename implementation
is very expensive.

In the preliminary work~\cite{betrfs4}, we introduce a new operation to \bets,
\rr, that accomplishes the work of file system renames with bounded IO cost.
A \rr slices out a source subtree from the \bet and moves the subtree to the
destination.
Unlike other write operations in \bets, \rr is not write-optimized and the
idea can be applied to other tree structures.

The idea of \rr can be extended to other namespace operations, like file clones.
A file clone creates a target file whose data are identical to the source file.
To clone a file, \betrfs can slice out the subtree as in \rr but share
the subtree between the source and the destination.

Many modern file systems are starting to support file clones.
Recently, Linux assigns a specific \texttt{ioctl} number for file clones
(FICLONE) and detects file clones in the virtual file system (VFS) layer for all
file systems.
Also, the reflink copy (\texttt{cp --reflink}) in Linux tries to invoke file
clones directly.

However, existing file clones struggle to deal with locality.
When one block of a cloned file is overwritten, the file system has to either
allocate an unused disk block, making the locality worse, or copy the whole
cloned extent to another place, resulting in large write amplification.

In contrast, the full-path indexing in \betrfs ensures locality and the
write-optimized \bets in \betrfs batch small overwrites to reduce write
amplification.

The thesis of this dissertation is: a full-path-indexed write-optimized file
system can have both good locality and efficient namespace operations by
implementing namespace operations as dictionary-internal operations.
We will demonstrate this thesis in the context of the \betrfs.

In the remainder of this proposal, Section~\ref{sec:bg} talks about the
background of \betrfs.
Section~\ref{sec:rename} shows how the rename problem was solved.
Section~\ref{sec:clone} proposes our solution to cloning.
Related works are shown in Section~\ref{sec:related}.
Section~\ref{sec:plan} presents the plan of milestones.
