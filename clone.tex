\chapter{Range-clone}
\label{chap:clone}

This chapter describes another \bet operation, range-clone, which can be used
to implement file or directory clones and renames on full-path-indexed \betrfs.
Unlikely a range-rename, which completes all its work at once, a range-clone
injects a new type of message, the \goto messages, to the \bet.
The \bet then flushes \goto messages with other messages in batches, gradually
finishing the range-clone works.
Therefore, range-clone fits into the write-optimized framework of \bets.

This chapter first describes the range-clone interface,
followed by how range-clones are implemented on \bets.
At last, this chapter describes a new technique, preferential splitting, which
makes range-clones more efficient.

\section{The range-clone interface}

\begin{table}[t]
    \centering
    \begin{tabular}{c | l}
        \hline
        Type of File System Operation & Key/Value Store Operations \\
        \hline
        \hline
        File Rename & \mdb$\rightarrow$put(\textit{dst}); \\
                    & \mdb$\rightarrow$del(\textit{src}); \\
                    & \ddb$\rightarrow$range-clone(\textit{src}, \textit{dst}, true); \\
        \hline
        Directory Rename & \mdb$\rightarrow$put(\textit{dst}); \\
                         & \mdb$\rightarrow$del(\textit{src}); \\
                         & \mdb$\rightarrow$range-clone(\textit{src/}, \textit{dst/}, true); \\
                         & \ddb$\rightarrow$range-clone(\textit{src/}, \textit{dst/}, true); \\
        \hline
        File Clone  & \mdb$\rightarrow$put(\textit{dst}); \\
                    & \mdb$\rightarrow$del(\textit{src}); \\
                    & \ddb$\rightarrow$range-clone(\textit{src}, \textit{dst}, false); \\
        \hline
        Directory Clone  & \mdb$\rightarrow$put(\textit{dst}); \\
                         & \mdb$\rightarrow$del(\textit{src}); \\
                         & \mdb$\rightarrow$range-clone(\textit{src/}, \textit{dst/}, false); \\
                         & \ddb$\rightarrow$range-clone(\textit{src/}, \textit{dst/}, false); \\
        \hline
    \end{tabular}
    \caption[File system renames and clones in \betrfs with range-clones]{\label{tab:fsrc}
        \betrfs renames or clones \textit{src} to \textit{dst} by calling range-clones.}
\end{table}

Simliar to range-rename, range-clone is defined as:
range-clone(\spre, \dpre, \delold), where \delold is a boolean.
It does the following things:
\begin{itemize}
\item it deletes all destination key/value pairs from the key/value store;
\item then, for any source key/value pair $(k,v)$ in the key/value store,
it inserts a key/value pair $(k',v)$ to the key/value store,
where $k$ is the concatenation of \spre and some suffix $s$ and $k'$ is the
concatenation of \dpre and the same suffix $s$;
\item if \delold is true, it deletes all source key/value pairs from the
key/value store.
\end{itemize}
If \delold is true, range-clone is equivalent to range-rename;
otherwise, range-clone is range-rename without deleting source key/value pairs.
Therefore, as summarized in Table~\ref{tab:fsrc},
file or directory renames on full-path-indexed \betrfs can be done
with range-clones just as they are done with range-renames.
And clones on full-path-index \betrfs are done in the same way with \delold set
to false.

One implementation of range-clone can utilize the range-rename code,
while making both the parents of LCAs have the source LCA as a child and
discarding the destination subtree,
instead of swapping and injecting a range-delete message.
This implementation also requires background garbage collection for discarded
subtrees and reference counting of nodes.

However, another goal of range-clone is to fit the operation into the
write-optimized framework of \bets.
Therefore, range-rename introduces a new type of messages, the \goto messages.

\section {Range-clone on \bets}

\section{Preferential splitting}
